% Options for packages loaded elsewhere
\PassOptionsToPackage{unicode}{hyperref}
\PassOptionsToPackage{hyphens}{url}
%
\documentclass[
]{article}
\usepackage{amsmath,amssymb}
\usepackage{iftex}
\ifPDFTeX
  \usepackage[T1]{fontenc}
  \usepackage[utf8]{inputenc}
  \usepackage{textcomp} % provide euro and other symbols
\else % if luatex or xetex
  \usepackage{unicode-math} % this also loads fontspec
  \defaultfontfeatures{Scale=MatchLowercase}
  \defaultfontfeatures[\rmfamily]{Ligatures=TeX,Scale=1}
\fi
\usepackage{lmodern}
\ifPDFTeX\else
  % xetex/luatex font selection
\fi
% Use upquote if available, for straight quotes in verbatim environments
\IfFileExists{upquote.sty}{\usepackage{upquote}}{}
\IfFileExists{microtype.sty}{% use microtype if available
  \usepackage[]{microtype}
  \UseMicrotypeSet[protrusion]{basicmath} % disable protrusion for tt fonts
}{}
\makeatletter
\@ifundefined{KOMAClassName}{% if non-KOMA class
  \IfFileExists{parskip.sty}{%
    \usepackage{parskip}
  }{% else
    \setlength{\parindent}{0pt}
    \setlength{\parskip}{6pt plus 2pt minus 1pt}}
}{% if KOMA class
  \KOMAoptions{parskip=half}}
\makeatother
\usepackage{xcolor}
\usepackage[margin=1in]{geometry}
\usepackage{color}
\usepackage{fancyvrb}
\newcommand{\VerbBar}{|}
\newcommand{\VERB}{\Verb[commandchars=\\\{\}]}
\DefineVerbatimEnvironment{Highlighting}{Verbatim}{commandchars=\\\{\}}
% Add ',fontsize=\small' for more characters per line
\usepackage{framed}
\definecolor{shadecolor}{RGB}{248,248,248}
\newenvironment{Shaded}{\begin{snugshade}}{\end{snugshade}}
\newcommand{\AlertTok}[1]{\textcolor[rgb]{0.94,0.16,0.16}{#1}}
\newcommand{\AnnotationTok}[1]{\textcolor[rgb]{0.56,0.35,0.01}{\textbf{\textit{#1}}}}
\newcommand{\AttributeTok}[1]{\textcolor[rgb]{0.13,0.29,0.53}{#1}}
\newcommand{\BaseNTok}[1]{\textcolor[rgb]{0.00,0.00,0.81}{#1}}
\newcommand{\BuiltInTok}[1]{#1}
\newcommand{\CharTok}[1]{\textcolor[rgb]{0.31,0.60,0.02}{#1}}
\newcommand{\CommentTok}[1]{\textcolor[rgb]{0.56,0.35,0.01}{\textit{#1}}}
\newcommand{\CommentVarTok}[1]{\textcolor[rgb]{0.56,0.35,0.01}{\textbf{\textit{#1}}}}
\newcommand{\ConstantTok}[1]{\textcolor[rgb]{0.56,0.35,0.01}{#1}}
\newcommand{\ControlFlowTok}[1]{\textcolor[rgb]{0.13,0.29,0.53}{\textbf{#1}}}
\newcommand{\DataTypeTok}[1]{\textcolor[rgb]{0.13,0.29,0.53}{#1}}
\newcommand{\DecValTok}[1]{\textcolor[rgb]{0.00,0.00,0.81}{#1}}
\newcommand{\DocumentationTok}[1]{\textcolor[rgb]{0.56,0.35,0.01}{\textbf{\textit{#1}}}}
\newcommand{\ErrorTok}[1]{\textcolor[rgb]{0.64,0.00,0.00}{\textbf{#1}}}
\newcommand{\ExtensionTok}[1]{#1}
\newcommand{\FloatTok}[1]{\textcolor[rgb]{0.00,0.00,0.81}{#1}}
\newcommand{\FunctionTok}[1]{\textcolor[rgb]{0.13,0.29,0.53}{\textbf{#1}}}
\newcommand{\ImportTok}[1]{#1}
\newcommand{\InformationTok}[1]{\textcolor[rgb]{0.56,0.35,0.01}{\textbf{\textit{#1}}}}
\newcommand{\KeywordTok}[1]{\textcolor[rgb]{0.13,0.29,0.53}{\textbf{#1}}}
\newcommand{\NormalTok}[1]{#1}
\newcommand{\OperatorTok}[1]{\textcolor[rgb]{0.81,0.36,0.00}{\textbf{#1}}}
\newcommand{\OtherTok}[1]{\textcolor[rgb]{0.56,0.35,0.01}{#1}}
\newcommand{\PreprocessorTok}[1]{\textcolor[rgb]{0.56,0.35,0.01}{\textit{#1}}}
\newcommand{\RegionMarkerTok}[1]{#1}
\newcommand{\SpecialCharTok}[1]{\textcolor[rgb]{0.81,0.36,0.00}{\textbf{#1}}}
\newcommand{\SpecialStringTok}[1]{\textcolor[rgb]{0.31,0.60,0.02}{#1}}
\newcommand{\StringTok}[1]{\textcolor[rgb]{0.31,0.60,0.02}{#1}}
\newcommand{\VariableTok}[1]{\textcolor[rgb]{0.00,0.00,0.00}{#1}}
\newcommand{\VerbatimStringTok}[1]{\textcolor[rgb]{0.31,0.60,0.02}{#1}}
\newcommand{\WarningTok}[1]{\textcolor[rgb]{0.56,0.35,0.01}{\textbf{\textit{#1}}}}
\usepackage{graphicx}
\makeatletter
\def\maxwidth{\ifdim\Gin@nat@width>\linewidth\linewidth\else\Gin@nat@width\fi}
\def\maxheight{\ifdim\Gin@nat@height>\textheight\textheight\else\Gin@nat@height\fi}
\makeatother
% Scale images if necessary, so that they will not overflow the page
% margins by default, and it is still possible to overwrite the defaults
% using explicit options in \includegraphics[width, height, ...]{}
\setkeys{Gin}{width=\maxwidth,height=\maxheight,keepaspectratio}
% Set default figure placement to htbp
\makeatletter
\def\fps@figure{htbp}
\makeatother
\setlength{\emergencystretch}{3em} % prevent overfull lines
\providecommand{\tightlist}{%
  \setlength{\itemsep}{0pt}\setlength{\parskip}{0pt}}
\setcounter{secnumdepth}{-\maxdimen} % remove section numbering
\ifLuaTeX
  \usepackage{selnolig}  % disable illegal ligatures
\fi
\usepackage{bookmark}
\IfFileExists{xurl.sty}{\usepackage{xurl}}{} % add URL line breaks if available
\urlstyle{same}
\hypersetup{
  pdftitle={1D-Bifurcations (Assignment Sheet 6)},
  pdfauthor={Alfonso Allen-Perkins, Juan Carlos Bueno and Eduardo Faleiro},
  hidelinks,
  pdfcreator={LaTeX via pandoc}}

\title{1D-Bifurcations (Assignment Sheet 6)}
\usepackage{etoolbox}
\makeatletter
\providecommand{\subtitle}[1]{% add subtitle to \maketitle
  \apptocmd{\@title}{\par {\large #1 \par}}{}{}
}
\makeatother
\subtitle{Introduction To Chaos Applied To Systems, Processes And
Products (ETSIDI, UPM)}
\author{Alfonso Allen-Perkins, Juan Carlos Bueno and Eduardo Faleiro}
\date{2025-04-03}

\begin{document}
\maketitle

{
\setcounter{tocdepth}{2}
\tableofcontents
}
\section{Introduction}\label{introduction}

Bifurcation theory describes how the qualitative behavior of equilibria
changes as a system parameter varies. Here we explore different types of
bifurcations using R.

Required Libraries

\begin{Shaded}
\begin{Highlighting}[]
\FunctionTok{library}\NormalTok{(ggplot2)}
\end{Highlighting}
\end{Shaded}

\begin{center}\rule{0.5\linewidth}{0.5pt}\end{center}

\section{1. Supercritical pitchfork
bifurcation}\label{supercritical-pitchfork-bifurcation}

\subsection{\texorpdfstring{\textbf{1.1 Theoretical
background}}{1.1 Theoretical background}}\label{theoretical-background}

A \textbf{supercritical pitchfork bifurcation} occurs when a stable
equilibrium splits into two new stable equilibria as the control
parameter \(r\) crosses a critical value. The system under study is:

\[
\begin{aligned}
\dot{x} &= r \cdot x - x^3
\end{aligned}
\]

The fixed points of this system are:

\[
x^* = 0, \quad \pm\sqrt{r}
\]

The number and stability of fixed points depend on the value of \(r\):

\begin{itemize}
\tightlist
\item
  \textbf{Case A}: \(r < 0\) → Only one stable fixed point at \(x = 0\).
\item
  \textbf{Case B}: \(r = 0\) → Bifurcation point where stability
  changes.
\item
  \textbf{Case C}: \(r > 0\) → The origin becomes unstable, and two new
  stable fixed points appear.
\end{itemize}

\subsection{\texorpdfstring{\textbf{1.2 Phase portraits for different
\(r\)}}{1.2 Phase portraits for different r}}\label{phase-portraits-for-different-r}

By plotting \(\dot{x}\) versus \(x\), we observe the change in stability
of the origin \(x^* = 0\). When \(r = -1\), the origin is stable; when
\(r = 1\), it becomes unstable.

\begin{Shaded}
\begin{Highlighting}[]
\NormalTok{supercritical\_map }\OtherTok{\textless{}{-}} \ControlFlowTok{function}\NormalTok{(x, r) \{}
  \FunctionTok{return}\NormalTok{(r }\SpecialCharTok{*}\NormalTok{ x }\SpecialCharTok{{-}}\NormalTok{ x}\SpecialCharTok{\^{}}\DecValTok{3}\NormalTok{)}
\NormalTok{\}}

\NormalTok{r\_values }\OtherTok{\textless{}{-}} \FunctionTok{c}\NormalTok{(}\SpecialCharTok{{-}}\DecValTok{1}\NormalTok{,}\DecValTok{0}\NormalTok{,}\DecValTok{1}\NormalTok{)}
\NormalTok{x\_vals }\OtherTok{\textless{}{-}} \FunctionTok{seq}\NormalTok{(}\SpecialCharTok{{-}}\FloatTok{1.5}\NormalTok{, }\FloatTok{1.5}\NormalTok{, }\AttributeTok{length.out =} \DecValTok{100}\NormalTok{)}
\NormalTok{dx\_dt\_df }\OtherTok{\textless{}{-}} \ConstantTok{NULL}  \CommentTok{\# Initialize as empty data frame}

\ControlFlowTok{for}\NormalTok{ (r }\ControlFlowTok{in}\NormalTok{ r\_values) \{}
\NormalTok{  df\_aux }\OtherTok{\textless{}{-}} \FunctionTok{data.frame}\NormalTok{(}
    \AttributeTok{x =}\NormalTok{ x\_vals,}
    \AttributeTok{dx\_dt =} \FunctionTok{supercritical\_map}\NormalTok{(x\_vals, r),}
    \AttributeTok{r =}\NormalTok{ r)}
\NormalTok{  dx\_dt\_df }\OtherTok{\textless{}{-}} \FunctionTok{rbind}\NormalTok{(dx\_dt\_df, df\_aux)  }\CommentTok{\# Append new rows}

\NormalTok{\}}

\FunctionTok{ggplot}\NormalTok{(}\AttributeTok{data =}\NormalTok{ dx\_dt\_df, }
       \FunctionTok{aes}\NormalTok{(}\AttributeTok{x =}\NormalTok{ x, }\AttributeTok{y =}\NormalTok{ dx\_dt, }\AttributeTok{color =} \FunctionTok{as.factor}\NormalTok{(r))) }\SpecialCharTok{+}
  \FunctionTok{geom\_line}\NormalTok{(}\AttributeTok{linewidth =} \DecValTok{1}\NormalTok{) }\SpecialCharTok{+}
  \FunctionTok{geom\_hline}\NormalTok{(}\AttributeTok{yintercept =} \DecValTok{0}\NormalTok{, }\AttributeTok{linetype =} \StringTok{"dashed"}\NormalTok{) }\SpecialCharTok{+}
  \FunctionTok{labs}\NormalTok{(}\AttributeTok{title =} \StringTok{"Supercritical pitchfork bifurcation for dx/dt = r x {-} x\^{}3"}\NormalTok{, }
       \AttributeTok{x =} \StringTok{"x"}\NormalTok{, }\AttributeTok{y =} \StringTok{"dx/dt"}\NormalTok{, }\AttributeTok{color =} \StringTok{"r"}\NormalTok{) }\SpecialCharTok{+}
  \FunctionTok{facet\_grid}\NormalTok{(}\SpecialCharTok{\textasciitilde{}}\NormalTok{r)}\SpecialCharTok{+}
  \FunctionTok{theme\_bw}\NormalTok{()}
\end{Highlighting}
\end{Shaded}

\includegraphics{6_AS_1D_Bifurcations_files/figure-latex/unnamed-chunk-2-1.pdf}

\subsection{\texorpdfstring{\textbf{1.3 Bifurcation
diagram}}{1.3 Bifurcation diagram}}\label{bifurcation-diagram}

To visualize how the nature of the roots depends on \(r\), we plot the
fixed points of the system against \(r\), using colors to indicate
stability:

\begin{itemize}
\tightlist
\item
  \textbf{Blue} = Stable equilibrium
\item
  \textbf{Red} = Unstable equilibrium
\end{itemize}

\begin{Shaded}
\begin{Highlighting}[]
\NormalTok{supercritical\_map }\OtherTok{\textless{}{-}} \ControlFlowTok{function}\NormalTok{(x, r) \{}
  \FunctionTok{return}\NormalTok{(r }\SpecialCharTok{*}\NormalTok{ x }\SpecialCharTok{{-}}\NormalTok{ x}\SpecialCharTok{\^{}}\DecValTok{3}\NormalTok{)}
\NormalTok{\}}

\NormalTok{fixed\_points\_supercritical }\OtherTok{\textless{}{-}} \ControlFlowTok{function}\NormalTok{(r) \{}
\NormalTok{  roots }\OtherTok{\textless{}{-}} \FunctionTok{polyroot}\NormalTok{(}\FunctionTok{c}\NormalTok{(}\DecValTok{0}\NormalTok{,r,}\DecValTok{0}\NormalTok{,}\SpecialCharTok{{-}}\DecValTok{1}\NormalTok{))}
\NormalTok{  real\_roots }\OtherTok{\textless{}{-}}\NormalTok{ roots[}\FunctionTok{round}\NormalTok{(}\FunctionTok{Im}\NormalTok{(roots),}\DecValTok{10}\NormalTok{)}\SpecialCharTok{==}\DecValTok{0}\NormalTok{]}
  \FunctionTok{return}\NormalTok{(}\FunctionTok{Re}\NormalTok{(real\_roots))}
\NormalTok{\}}

\NormalTok{stability\_supercritical }\OtherTok{\textless{}{-}} \ControlFlowTok{function}\NormalTok{(x, r) \{}
\NormalTok{  derivative }\OtherTok{\textless{}{-}}\NormalTok{ r }\SpecialCharTok{{-}} \DecValTok{3} \SpecialCharTok{*}\NormalTok{ x}\SpecialCharTok{\^{}}\DecValTok{2}
  \ControlFlowTok{if}\NormalTok{ (derivative }\SpecialCharTok{\textless{}} \DecValTok{0}\NormalTok{) }\StringTok{"Stable"} \ControlFlowTok{else} \StringTok{"Unstable"}
\NormalTok{\}}

\NormalTok{r\_values }\OtherTok{\textless{}{-}} \FunctionTok{seq}\NormalTok{(}\SpecialCharTok{{-}}\DecValTok{2}\NormalTok{, }\DecValTok{2}\NormalTok{, }\AttributeTok{length.out =} \DecValTok{500}\NormalTok{)}
\NormalTok{bifurcation\_data }\OtherTok{\textless{}{-}} \ConstantTok{NULL} \CommentTok{\#Empty variable to store bifurcation data}

\ControlFlowTok{for}\NormalTok{ (r }\ControlFlowTok{in}\NormalTok{ r\_values) \{}
\NormalTok{  points\_super }\OtherTok{\textless{}{-}} \FunctionTok{fixed\_points\_supercritical}\NormalTok{(r)}
  \ControlFlowTok{for}\NormalTok{ (x }\ControlFlowTok{in}\NormalTok{ points\_super) \{}
\NormalTok{    bifurcation\_data }\OtherTok{\textless{}{-}} \FunctionTok{rbind}\NormalTok{(bifurcation\_data, }
                              \FunctionTok{data.frame}\NormalTok{(}\AttributeTok{r =}\NormalTok{ r, }
                                         \AttributeTok{x =}\NormalTok{ x, }
                                         \AttributeTok{stability =} \FunctionTok{stability\_supercritical}\NormalTok{(x, r)}
\NormalTok{                                         )}
\NormalTok{                              )}
\NormalTok{  \}}
\NormalTok{\}}

\FunctionTok{ggplot}\NormalTok{(bifurcation\_data, }\FunctionTok{aes}\NormalTok{(}\AttributeTok{x =}\NormalTok{ r, }\AttributeTok{y =}\NormalTok{ x, }\AttributeTok{color =}\NormalTok{ stability)) }\SpecialCharTok{+}
  \FunctionTok{geom\_point}\NormalTok{(}\AttributeTok{size =} \FloatTok{1.5}\NormalTok{, }\AttributeTok{alpha =} \FloatTok{0.7}\NormalTok{) }\SpecialCharTok{+}
  \FunctionTok{scale\_color\_manual}\NormalTok{(}\AttributeTok{values =} \FunctionTok{c}\NormalTok{(}\StringTok{"Stable"} \OtherTok{=} \StringTok{"blue"}\NormalTok{, }\StringTok{"Unstable"} \OtherTok{=} \StringTok{"red"}\NormalTok{)) }\SpecialCharTok{+}
  \FunctionTok{labs}\NormalTok{(}\AttributeTok{title =} \StringTok{"Supercritical pitchfork bifurcation diagram"}\NormalTok{,}
       \AttributeTok{x =} \StringTok{"r (Control parameter)"}\NormalTok{, }\AttributeTok{y =} \StringTok{"Fixed Points"}\NormalTok{) }\SpecialCharTok{+}
  \FunctionTok{theme\_minimal}\NormalTok{()}
\end{Highlighting}
\end{Shaded}

\includegraphics{6_AS_1D_Bifurcations_files/figure-latex/unnamed-chunk-3-1.pdf}

\begin{center}\rule{0.5\linewidth}{0.5pt}\end{center}

\section{2. Subcritical pitchfork
bifurcation}\label{subcritical-pitchfork-bifurcation}

\subsection{\texorpdfstring{\textbf{2.1 Theoretical
background}}{2.1 Theoretical background}}\label{theoretical-background-1}

A \textbf{subcritical pitchfork bifurcation} is characterized by an
unstable fixed point that merges with a stable one as the control
parameter \(r\) changes. The system under study is:

\[
\begin{aligned}
\dot{x} &= r \cdot x + x^3
\end{aligned}
\]

The fixed points of this system are:

\[
x^* = 0, \quad \pm\sqrt{-r}
\]

\subsection{\texorpdfstring{\textbf{2.2 Bifurcation
diagram}}{2.2 Bifurcation diagram}}\label{bifurcation-diagram-1}

To visualize how the nature of the roots depends on \(r\), we plot the
fixed points of the system against \(r\), using colors to indicate
stability:

\begin{itemize}
\tightlist
\item
  \textbf{Blue} = Stable equilibrium
\item
  \textbf{Red} = Unstable equilibrium
\end{itemize}

\begin{Shaded}
\begin{Highlighting}[]
\NormalTok{subcritical\_map }\OtherTok{\textless{}{-}} \ControlFlowTok{function}\NormalTok{(x, r) \{}
  \FunctionTok{return}\NormalTok{(r }\SpecialCharTok{*}\NormalTok{ x }\SpecialCharTok{+}\NormalTok{ x}\SpecialCharTok{\^{}}\DecValTok{3}\NormalTok{)}
\NormalTok{\}}

\NormalTok{fixed\_points\_subcritical }\OtherTok{\textless{}{-}} \ControlFlowTok{function}\NormalTok{(r) \{}
\NormalTok{  roots }\OtherTok{\textless{}{-}} \FunctionTok{polyroot}\NormalTok{(}\FunctionTok{c}\NormalTok{(}\DecValTok{0}\NormalTok{,r,}\DecValTok{0}\NormalTok{,}\DecValTok{1}\NormalTok{))}
\NormalTok{  real\_roots }\OtherTok{\textless{}{-}}\NormalTok{ roots[}\FunctionTok{round}\NormalTok{(}\FunctionTok{Im}\NormalTok{(roots),}\DecValTok{10}\NormalTok{)}\SpecialCharTok{==}\DecValTok{0}\NormalTok{]}
  \FunctionTok{return}\NormalTok{(}\FunctionTok{Re}\NormalTok{(real\_roots))}
\NormalTok{\}}

\NormalTok{stability\_subcritical }\OtherTok{\textless{}{-}} \ControlFlowTok{function}\NormalTok{(x, r) \{}
\NormalTok{  derivative }\OtherTok{\textless{}{-}}\NormalTok{ r }\SpecialCharTok{+} \DecValTok{3} \SpecialCharTok{*}\NormalTok{ x}\SpecialCharTok{\^{}}\DecValTok{2}
  \ControlFlowTok{if}\NormalTok{ (derivative }\SpecialCharTok{\textless{}} \DecValTok{0}\NormalTok{) }\StringTok{"Stable"} \ControlFlowTok{else} \StringTok{"Unstable"}
\NormalTok{\}}

\NormalTok{r\_values }\OtherTok{\textless{}{-}} \FunctionTok{seq}\NormalTok{(}\SpecialCharTok{{-}}\DecValTok{2}\NormalTok{, }\DecValTok{2}\NormalTok{, }\AttributeTok{length.out =} \DecValTok{500}\NormalTok{)}
\NormalTok{bifurcation\_data }\OtherTok{\textless{}{-}} \ConstantTok{NULL} \CommentTok{\#Empty variable to store bifurcation data}

\ControlFlowTok{for}\NormalTok{ (r }\ControlFlowTok{in}\NormalTok{ r\_values) \{}
\NormalTok{  points\_sub }\OtherTok{\textless{}{-}} \FunctionTok{fixed\_points\_subcritical}\NormalTok{(r)}
  \ControlFlowTok{for}\NormalTok{ (x }\ControlFlowTok{in}\NormalTok{ points\_sub) \{}
\NormalTok{    bifurcation\_data }\OtherTok{\textless{}{-}} \FunctionTok{rbind}\NormalTok{(bifurcation\_data, }
                              \FunctionTok{data.frame}\NormalTok{(}\AttributeTok{r =}\NormalTok{ r, }
                                         \AttributeTok{x =}\NormalTok{ x, }
                                         \AttributeTok{stability =} \FunctionTok{stability\_subcritical}\NormalTok{(x, r)}
\NormalTok{                                         )}
\NormalTok{                              )}
\NormalTok{  \}}
\NormalTok{\}}

\FunctionTok{ggplot}\NormalTok{(bifurcation\_data, }\FunctionTok{aes}\NormalTok{(}\AttributeTok{x =}\NormalTok{ r, }\AttributeTok{y =}\NormalTok{ x, }\AttributeTok{color =}\NormalTok{ stability)) }\SpecialCharTok{+}
  \FunctionTok{geom\_point}\NormalTok{(}\AttributeTok{size =} \FloatTok{1.5}\NormalTok{, }\AttributeTok{alpha =} \FloatTok{0.7}\NormalTok{) }\SpecialCharTok{+}
  \FunctionTok{scale\_color\_manual}\NormalTok{(}\AttributeTok{values =} \FunctionTok{c}\NormalTok{(}\StringTok{"Stable"} \OtherTok{=} \StringTok{"blue"}\NormalTok{, }\StringTok{"Unstable"} \OtherTok{=} \StringTok{"red"}\NormalTok{)) }\SpecialCharTok{+}
  \FunctionTok{labs}\NormalTok{(}\AttributeTok{title =} \StringTok{"Subcritical pitchfork bifurcation diagram"}\NormalTok{, }
       \AttributeTok{x =} \StringTok{"r (Control parameter)"}\NormalTok{, }\AttributeTok{y =} \StringTok{"Fixed points"}\NormalTok{) }\SpecialCharTok{+}
  \FunctionTok{theme\_bw}\NormalTok{()}
\end{Highlighting}
\end{Shaded}

\includegraphics{6_AS_1D_Bifurcations_files/figure-latex/subcritical_bifurcation-1.pdf}

\begin{center}\rule{0.5\linewidth}{0.5pt}\end{center}

\section{3. Saddle-Node bifurcation}\label{saddle-node-bifurcation}

\subsection{\texorpdfstring{\textbf{3.1 Theoretical
background}}{3.1 Theoretical background}}\label{theoretical-background-2}

A \textbf{saddle-Node bifurcation} occurs when two fixed points (one
stable and one unstable) collide and annihilate each other as the
control parameter \(r\) crosses a critical value. Consider the system:

\[
\begin{aligned}
\dot{x} &= r  - x^2
\end{aligned}
\]

The fixed points of this system are:

\[
x^*  \pm\sqrt{r}
\]

Depending on the value of \(r\):

\begin{itemize}
\tightlist
\item
  \textbf{Case A}: \(r < 0\) → No real fixed points, so the system does
  not have equilibrium solutions..
\item
  \textbf{Case B}: \(r = 0\) → A single real fixed point at \(x^*=0\).
  This is the point at which two fixed points collide.
\item
  \textbf{Case C}: \(r > 0\) → Two real fixed points
  appear:\(x^*= + \sqrt{r}\) and \(x^*= - \sqrt{r}\). One will be
  stable, and the other unstable, reflecting the ``saddle'' (unstable)
  and ``node'' (stable) nature.
\end{itemize}

The stability can be determined by evaluating the derivative of the
right-hand side:

\[
\frac{d }{dt}(r-x^2)=-2\cdot x
\]

\begin{itemize}
\item
  If \textbf{\(-2\cdot x<0\)}, the fixed point is \textbf{stable}.
\item
  If \textbf{\(-2\cdot x>0\)}, the fixed point is \textbf{unstable}.
\end{itemize}

\subsection{\texorpdfstring{\textbf{3.2 Phase portraits for different
\(r\)}}{3.2 Phase portraits for different r}}\label{phase-portraits-for-different-r-1}

Similar to the supercritical case, we can plot \(\dot{x} = r  - x^2\) as
a function of \(x\) for representative values of \(r\). Note how the
number of intersections with \(\dot{x} =0\) changes from no intersection
(when \(r < 0\)) to one intersection at \(r = 0\), and two intersections
(one stable, one unstable) when \(r > 0\).

By plotting \(\dot{x}\) versus \(x\), we observe the change in stability
of the origin \(x^* = 0\). When \(r = -1\), the origin is stable; when
\(r = 1\), it becomes unstable.

\begin{Shaded}
\begin{Highlighting}[]
\CommentTok{\# Phase portrait function}
\NormalTok{saddle\_node\_map }\OtherTok{\textless{}{-}} \ControlFlowTok{function}\NormalTok{(x, r) \{}
  \FunctionTok{return}\NormalTok{(r }\SpecialCharTok{{-}}\NormalTok{ x}\SpecialCharTok{\^{}}\DecValTok{2}\NormalTok{)}
\NormalTok{\}}

\CommentTok{\# Choose representative values of r}
\NormalTok{r\_values }\OtherTok{\textless{}{-}} \FunctionTok{c}\NormalTok{(}\SpecialCharTok{{-}}\FloatTok{0.5}\NormalTok{, }\DecValTok{0}\NormalTok{, }\DecValTok{1}\NormalTok{)}
\NormalTok{x\_vals }\OtherTok{\textless{}{-}} \FunctionTok{seq}\NormalTok{(}\SpecialCharTok{{-}}\DecValTok{2}\NormalTok{, }\DecValTok{2}\NormalTok{, }\AttributeTok{length.out =} \DecValTok{200}\NormalTok{)}

\CommentTok{\# Build data frame for plotting}
\NormalTok{df\_phase }\OtherTok{\textless{}{-}} \FunctionTok{data.frame}\NormalTok{()}

\ControlFlowTok{for}\NormalTok{ (r }\ControlFlowTok{in}\NormalTok{ r\_values) \{}
\NormalTok{  df\_temp }\OtherTok{\textless{}{-}} \FunctionTok{data.frame}\NormalTok{(}
    \AttributeTok{x =}\NormalTok{ x\_vals,}
    \AttributeTok{dx\_dt =} \FunctionTok{saddle\_node\_map}\NormalTok{(x\_vals, r),}
    \AttributeTok{r =} \FunctionTok{as.factor}\NormalTok{(r)}
\NormalTok{  )}
\NormalTok{  df\_phase }\OtherTok{\textless{}{-}} \FunctionTok{rbind}\NormalTok{(df\_phase, df\_temp)}
\NormalTok{\}}

\FunctionTok{library}\NormalTok{(ggplot2)}

\FunctionTok{ggplot}\NormalTok{(df\_phase, }\FunctionTok{aes}\NormalTok{(}\AttributeTok{x =}\NormalTok{ x, }\AttributeTok{y =}\NormalTok{ dx\_dt, }\AttributeTok{color =}\NormalTok{ r)) }\SpecialCharTok{+}
  \FunctionTok{geom\_line}\NormalTok{(}\AttributeTok{linewidth =} \DecValTok{1}\NormalTok{) }\SpecialCharTok{+}
  \FunctionTok{geom\_vline}\NormalTok{(}\AttributeTok{xintercept =} \DecValTok{0}\NormalTok{, }\AttributeTok{linetype =} \StringTok{"dashed"}\NormalTok{) }\SpecialCharTok{+}
  \FunctionTok{geom\_hline}\NormalTok{(}\AttributeTok{yintercept =} \DecValTok{0}\NormalTok{, }\AttributeTok{linetype =} \StringTok{"dashed"}\NormalTok{) }\SpecialCharTok{+}
  \FunctionTok{labs}\NormalTok{(}\AttributeTok{title =} \StringTok{"Saddle{-}node bifurcation: Phase portraits"}\NormalTok{,}
       \AttributeTok{x =} \StringTok{"x"}\NormalTok{, }\AttributeTok{y =} \StringTok{"dx/dt"}\NormalTok{, }\AttributeTok{color =} \StringTok{"r"}\NormalTok{) }\SpecialCharTok{+}
  \FunctionTok{facet\_grid}\NormalTok{(}\SpecialCharTok{\textasciitilde{}}\NormalTok{r) }\SpecialCharTok{+}
  \FunctionTok{theme\_bw}\NormalTok{()}
\end{Highlighting}
\end{Shaded}

\includegraphics{6_AS_1D_Bifurcations_files/figure-latex/unnamed-chunk-4-1.pdf}

\begin{itemize}
\tightlist
\item
  For \(r =-0.5\), there are no real fixed points (the curve never
  crosses the horizontal line \(\dot{x} =0\).
\item
  For \(r =0\), there is one real fixed point at \(x =0\) (the curve is
  tangent to the horizontal axis).
\item
  For \(r = 1\), there are two real fixed points, \(x =\pm 1\). One is
  stable ( \(x<0\) since \(-2\cdot x >0\) there), and the other is
  unstable ( \(x>0\) since since \(-2\cdot x < 0\) there).
\end{itemize}

\subsection{\texorpdfstring{\textbf{3.3 Bifurcation
diagram}}{3.3 Bifurcation diagram}}\label{bifurcation-diagram-2}

Finally, we construct the bifurcation diagram by identifying the fixed
points across a range of \(r\) values and coloring them according to
their stability:

\begin{itemize}
\tightlist
\item
  \textbf{Blue} = Stable equilibrium
\item
  \textbf{Red} = Unstable equilibrium
\end{itemize}

Notice how, for \(r > 0\), two branches appear (stable and unstable),
and for \(r < 0\), there is no real fixed point.

\begin{Shaded}
\begin{Highlighting}[]
\CommentTok{\# map\_function \textless{}{-} function(x, r) \{}
\CommentTok{\#   return(r {-} x\^{}2)}
\CommentTok{\# \}}

\NormalTok{fixed\_points\_saddle\_node }\OtherTok{\textless{}{-}} \ControlFlowTok{function}\NormalTok{(r) \{}
\NormalTok{  roots }\OtherTok{\textless{}{-}} \FunctionTok{polyroot}\NormalTok{(}\FunctionTok{c}\NormalTok{(r,}\DecValTok{0}\NormalTok{,}\SpecialCharTok{{-}}\DecValTok{1}\NormalTok{))}
\NormalTok{  real\_roots }\OtherTok{\textless{}{-}}\NormalTok{ roots[}\FunctionTok{round}\NormalTok{(}\FunctionTok{Im}\NormalTok{(roots),}\DecValTok{10}\NormalTok{)}\SpecialCharTok{==}\DecValTok{0}\NormalTok{]}
  \FunctionTok{return}\NormalTok{(}\FunctionTok{Re}\NormalTok{(real\_roots))}
\NormalTok{\}}

\NormalTok{stability\_saddle\_node }\OtherTok{\textless{}{-}} \ControlFlowTok{function}\NormalTok{(x, r) \{}
\NormalTok{  derivative }\OtherTok{\textless{}{-}} \SpecialCharTok{{-}}\DecValTok{2} \SpecialCharTok{*}\NormalTok{ x}
  \ControlFlowTok{if}\NormalTok{ (derivative }\SpecialCharTok{\textless{}} \DecValTok{0}\NormalTok{) }\StringTok{"Stable"} \ControlFlowTok{else} \StringTok{"Unstable"}
\NormalTok{\}}

\NormalTok{r\_values }\OtherTok{\textless{}{-}} \FunctionTok{seq}\NormalTok{(}\SpecialCharTok{{-}}\DecValTok{1}\NormalTok{, }\DecValTok{2}\NormalTok{, }\AttributeTok{length.out =} \DecValTok{500}\NormalTok{)}
\NormalTok{bifurcation\_data }\OtherTok{\textless{}{-}} \ConstantTok{NULL} \CommentTok{\#Empty variable to store bifurcation data}

\ControlFlowTok{for}\NormalTok{ (r }\ControlFlowTok{in}\NormalTok{ r\_values) \{}
\NormalTok{  points }\OtherTok{\textless{}{-}} \FunctionTok{fixed\_points\_saddle\_node}\NormalTok{(r)}
  \ControlFlowTok{for}\NormalTok{ (x }\ControlFlowTok{in}\NormalTok{ points) \{}
\NormalTok{    bifurcation\_data }\OtherTok{\textless{}{-}} \FunctionTok{rbind}\NormalTok{(bifurcation\_data, }
                              \FunctionTok{data.frame}\NormalTok{(}\AttributeTok{r =}\NormalTok{ r, }
                                         \AttributeTok{x =}\NormalTok{ x, }
                                         \AttributeTok{stability =} \FunctionTok{stability\_saddle\_node}\NormalTok{(x, r)}
\NormalTok{                                         )}
\NormalTok{                              )}
\NormalTok{  \}}
\NormalTok{\}}

\FunctionTok{ggplot}\NormalTok{(bifurcation\_data, }\FunctionTok{aes}\NormalTok{(}\AttributeTok{x =}\NormalTok{ r, }\AttributeTok{y =}\NormalTok{ x, }\AttributeTok{color =}\NormalTok{ stability)) }\SpecialCharTok{+}
  \FunctionTok{geom\_point}\NormalTok{(}\AttributeTok{size =} \FloatTok{1.5}\NormalTok{, }\AttributeTok{alpha =} \FloatTok{0.7}\NormalTok{) }\SpecialCharTok{+}
  \FunctionTok{scale\_color\_manual}\NormalTok{(}\AttributeTok{values =} \FunctionTok{c}\NormalTok{(}\StringTok{"Stable"} \OtherTok{=} \StringTok{"blue"}\NormalTok{, }\StringTok{"Unstable"} \OtherTok{=} \StringTok{"red"}\NormalTok{)) }\SpecialCharTok{+}
  \FunctionTok{labs}\NormalTok{(}\AttributeTok{title =} \StringTok{"Saddle{-}node bifurcation diagram"}\NormalTok{, }
       \AttributeTok{x =} \StringTok{"r (Control parameter)"}\NormalTok{, }\AttributeTok{y =} \StringTok{"Fixed points"}\NormalTok{) }\SpecialCharTok{+}
  \FunctionTok{theme\_bw}\NormalTok{()}
\end{Highlighting}
\end{Shaded}

\includegraphics{6_AS_1D_Bifurcations_files/figure-latex/saddle_node_bifurcation-1.pdf}

\begin{center}\rule{0.5\linewidth}{0.5pt}\end{center}

\section{4. Transcritical Bifurcation}\label{transcritical-bifurcation}

\subsection{\texorpdfstring{\textbf{4.1 Theoretical
background}}{4.1 Theoretical background}}\label{theoretical-background-3}

A \textbf{transcritical bifurcation} occurs when two fixed points
exchange their stability as a parameter is varied. Unlike the
saddle-node bifurcation, in the transcritical case both fixed points
exist for all parameter values, but their stability changes at the
critical point. Consider the system:

Consider the system:

\[
\dot{x} = rx - x^2
\]

This system has two fixed points:

\begin{itemize}
\tightlist
\item
  \(x^* = 0\)
\item
  \(x^* = r\)
\end{itemize}

The stability of fixed points depend on the value of \(r\):

\begin{itemize}
\tightlist
\item
  At \(x^* = 0\): \(\left.\frac{d}{dx}(rx - x^2)\right|_{x^*=0} = r\)
\item
  At \(x^* = r\): \(\left.\frac{d}{dx}(rx - x^2)\right|_{x^*=r} = -r\)
\end{itemize}

Therefore:

\begin{itemize}
\tightlist
\item
  If \(r < 0\), \(x^*=0\) is stable and \(x^*=r\) is unstable.
\item
  If \(r > 0\), \(x^*=0\) is unstable and \(x^*=r\) is stable.
\item
  At \(r=0\), both fixed points coincide and exchange stability.
\end{itemize}

\subsection{\texorpdfstring{\textbf{4.2 Phase portraits for different
\(r\)}}{4.2 Phase portraits for different r}}\label{phase-portraits-for-different-r-2}

By plotting \(\dot{x}\) versus \(x\), we observe the change in stability
of the fixed points \(x^*=0\) and \(x^*=r\). When \(r = -1\), the origin
is stable; when \(r = 1\), the roles are reversed: the origin becomes
unstable and the other fixed point is stable.

\begin{Shaded}
\begin{Highlighting}[]
\NormalTok{transcritical\_map }\OtherTok{\textless{}{-}} \ControlFlowTok{function}\NormalTok{(x, r) \{}
  \FunctionTok{return}\NormalTok{(r}\SpecialCharTok{*}\NormalTok{x  }\SpecialCharTok{{-}}\NormalTok{ x}\SpecialCharTok{\^{}}\DecValTok{2}\NormalTok{)}
\NormalTok{\}}

\NormalTok{r\_values }\OtherTok{\textless{}{-}} \FunctionTok{c}\NormalTok{(}\SpecialCharTok{{-}}\DecValTok{1}\NormalTok{,}\DecValTok{0}\NormalTok{,}\DecValTok{1}\NormalTok{)}
\NormalTok{x\_vals }\OtherTok{\textless{}{-}} \FunctionTok{seq}\NormalTok{(}\SpecialCharTok{{-}}\FloatTok{1.5}\NormalTok{, }\FloatTok{1.5}\NormalTok{, }\AttributeTok{length.out =} \DecValTok{100}\NormalTok{)}
\NormalTok{dx\_dt\_df }\OtherTok{\textless{}{-}} \ConstantTok{NULL}  \CommentTok{\# Initialize as empty data frame}

\ControlFlowTok{for}\NormalTok{ (r }\ControlFlowTok{in}\NormalTok{ r\_values) \{}
\NormalTok{  df\_aux }\OtherTok{\textless{}{-}} \FunctionTok{data.frame}\NormalTok{(}
    \AttributeTok{x =}\NormalTok{ x\_vals,}
    \AttributeTok{dx\_dt =} \FunctionTok{transcritical\_map}\NormalTok{(x\_vals, r),}
    \AttributeTok{r =}\NormalTok{ r)}
\NormalTok{  dx\_dt\_df }\OtherTok{\textless{}{-}} \FunctionTok{rbind}\NormalTok{(dx\_dt\_df, df\_aux)  }\CommentTok{\# Append new rows}

\NormalTok{\}}

\FunctionTok{ggplot}\NormalTok{(}\AttributeTok{data =}\NormalTok{ dx\_dt\_df, }
       \FunctionTok{aes}\NormalTok{(}\AttributeTok{x =}\NormalTok{ x, }\AttributeTok{y =}\NormalTok{ dx\_dt, }\AttributeTok{color =} \FunctionTok{as.factor}\NormalTok{(r))) }\SpecialCharTok{+}
  \FunctionTok{geom\_line}\NormalTok{(}\AttributeTok{linewidth =} \DecValTok{1}\NormalTok{) }\SpecialCharTok{+}
  \FunctionTok{geom\_vline}\NormalTok{(}\AttributeTok{xintercept =} \DecValTok{0}\NormalTok{, }\AttributeTok{linetype =} \StringTok{"dashed"}\NormalTok{) }\SpecialCharTok{+}
  \FunctionTok{geom\_hline}\NormalTok{(}\AttributeTok{yintercept =} \DecValTok{0}\NormalTok{, }\AttributeTok{linetype =} \StringTok{"dashed"}\NormalTok{) }\SpecialCharTok{+}
  \FunctionTok{labs}\NormalTok{(}\AttributeTok{title =} \StringTok{"Transcritical bifurcation for dx/dt = rx {-} x\^{}2"}\NormalTok{, }
       \AttributeTok{x =} \StringTok{"x"}\NormalTok{, }\AttributeTok{y =} \StringTok{"dx/dt"}\NormalTok{, }\AttributeTok{color =} \StringTok{"r"}\NormalTok{) }\SpecialCharTok{+}
  \FunctionTok{facet\_grid}\NormalTok{(}\SpecialCharTok{\textasciitilde{}}\NormalTok{r)}\SpecialCharTok{+}
  \FunctionTok{theme\_bw}\NormalTok{()}
\end{Highlighting}
\end{Shaded}

\includegraphics{6_AS_1D_Bifurcations_files/figure-latex/unnamed-chunk-5-1.pdf}

\subsection{\texorpdfstring{\textbf{4.3 Bifurcation
diagram}}{4.3 Bifurcation diagram}}\label{bifurcation-diagram-3}

Finally, we build the bifurcation diagram by tracking the fixed points
as the parameter \(r\) varies, and we color-code them based on their
stability:

\begin{itemize}
\tightlist
\item
  \textbf{Blue} = Stable equilibrium
\item
  \textbf{Red} = Unstable equilibrium
\end{itemize}

\begin{Shaded}
\begin{Highlighting}[]
\CommentTok{\# map\_function \textless{}{-} function(x, r) \{}
\CommentTok{\#   return(r*x {-} x\^{}2)}
\CommentTok{\# \}}

\NormalTok{fixed\_points\_transcritical\_map }\OtherTok{\textless{}{-}} \ControlFlowTok{function}\NormalTok{(r) \{}
\NormalTok{  roots }\OtherTok{\textless{}{-}} \FunctionTok{polyroot}\NormalTok{(}\FunctionTok{c}\NormalTok{(}\DecValTok{0}\NormalTok{,r,}\SpecialCharTok{{-}}\DecValTok{1}\NormalTok{))}
\NormalTok{  real\_roots }\OtherTok{\textless{}{-}}\NormalTok{ roots[}\FunctionTok{round}\NormalTok{(}\FunctionTok{Im}\NormalTok{(roots),}\DecValTok{10}\NormalTok{)}\SpecialCharTok{==}\DecValTok{0}\NormalTok{]}
  \FunctionTok{return}\NormalTok{(}\FunctionTok{Re}\NormalTok{(real\_roots))}
\NormalTok{\}}

\NormalTok{stability\_transcritical\_map }\OtherTok{\textless{}{-}} \ControlFlowTok{function}\NormalTok{(x, r) \{}
\NormalTok{  derivative }\OtherTok{\textless{}{-}}\NormalTok{ r }\SpecialCharTok{{-}}\DecValTok{2} \SpecialCharTok{*}\NormalTok{ x}
  \ControlFlowTok{if}\NormalTok{ (derivative }\SpecialCharTok{\textless{}} \DecValTok{0}\NormalTok{) }\StringTok{"Stable"} \ControlFlowTok{else} \StringTok{"Unstable"}
\NormalTok{\}}

\NormalTok{r\_values }\OtherTok{\textless{}{-}} \FunctionTok{seq}\NormalTok{(}\SpecialCharTok{{-}}\DecValTok{1}\NormalTok{, }\DecValTok{2}\NormalTok{, }\AttributeTok{length.out =} \DecValTok{500}\NormalTok{)}
\NormalTok{bifurcation\_data }\OtherTok{\textless{}{-}} \ConstantTok{NULL} \CommentTok{\#Empty variable to store bifurcation data}

\ControlFlowTok{for}\NormalTok{ (r }\ControlFlowTok{in}\NormalTok{ r\_values) \{}
\NormalTok{  points }\OtherTok{\textless{}{-}} \FunctionTok{fixed\_points\_transcritical\_map}\NormalTok{(r)}
  \ControlFlowTok{for}\NormalTok{ (x }\ControlFlowTok{in}\NormalTok{ points) \{}
\NormalTok{    bifurcation\_data }\OtherTok{\textless{}{-}} \FunctionTok{rbind}\NormalTok{(bifurcation\_data, }
                              \FunctionTok{data.frame}\NormalTok{(}\AttributeTok{r =}\NormalTok{ r, }
                                         \AttributeTok{x =}\NormalTok{ x, }
                                         \AttributeTok{stability =} \FunctionTok{stability\_transcritical\_map}\NormalTok{(x, r)}
\NormalTok{                                         )}
\NormalTok{                              )}
\NormalTok{  \}}
\NormalTok{\}}

\FunctionTok{ggplot}\NormalTok{(bifurcation\_data, }\FunctionTok{aes}\NormalTok{(}\AttributeTok{x =}\NormalTok{ r, }\AttributeTok{y =}\NormalTok{ x, }\AttributeTok{color =}\NormalTok{ stability)) }\SpecialCharTok{+}
  \FunctionTok{geom\_point}\NormalTok{(}\AttributeTok{size =} \FloatTok{1.5}\NormalTok{, }\AttributeTok{alpha =} \FloatTok{0.7}\NormalTok{) }\SpecialCharTok{+}
  \FunctionTok{scale\_color\_manual}\NormalTok{(}\AttributeTok{values =} \FunctionTok{c}\NormalTok{(}\StringTok{"Stable"} \OtherTok{=} \StringTok{"blue"}\NormalTok{, }\StringTok{"Unstable"} \OtherTok{=} \StringTok{"red"}\NormalTok{)) }\SpecialCharTok{+}
  \FunctionTok{labs}\NormalTok{(}\AttributeTok{title =} \StringTok{"Transcritical bifurcation diagram"}\NormalTok{, }
       \AttributeTok{x =} \StringTok{"r (Control parameter)"}\NormalTok{, }\AttributeTok{y =} \StringTok{"Fixed points"}\NormalTok{) }\SpecialCharTok{+}
  \FunctionTok{theme\_bw}\NormalTok{()}
\end{Highlighting}
\end{Shaded}

\includegraphics{6_AS_1D_Bifurcations_files/figure-latex/unnamed-chunk-6-1.pdf}

\begin{center}\rule{0.5\linewidth}{0.5pt}\end{center}

\section{Exercise}\label{exercise}

This pactice exercise is focused on characterizing the
\textbf{subcritical pitchfork bifurcation} of the system:

\[
\dot{x} = r\,x + x^3 - x^5.
\]

\begin{enumerate}
\def\labelenumi{\arabic{enumi}.}
\tightlist
\item
  \textbf{Find the Equilibrium Points}

  \begin{itemize}
  \tightlist
  \item
    Write down the equilibrium equation \(\dot{x} = 0\).\\
  \item
    Factorize the polynomial to identify the solutions for \(x\) and see
    how they depend on \(r\).
  \end{itemize}
\item
  \textbf{Stability Analysis}

  \begin{itemize}
  \tightlist
  \item
    For each equilibrium \(x^*\), compute the derivative
    \(\frac{d}{dx}(r\,x + x^3 - x^5) = r + 3x^2 - 5x^4\).\\
  \item
    Conclude whether the equilibrium is \textbf{stable} (derivative
    \textless{} 0) or \textbf{unstable} (derivative \textgreater{} 0).
  \end{itemize}
\item
  \textbf{Sketch Phase Portraits}

  \begin{itemize}
  \tightlist
  \item
    Choose representative values of \(r\) (negative, zero, positive).\\
  \item
    Plot \(\dot{x}\) vs.~\(x\) in R, identify where \(\dot{x} = 0\), and
    mark equilibria as stable or unstable.
  \end{itemize}
\item
  \textbf{Numerical Bifurcation Diagram}

  \begin{itemize}
  \tightlist
  \item
    Adapt or run the R script above, which uses polynomial root-finding
    (\texttt{polyroot}) to locate equilibria.\\
  \item
    Color them by stability and plot them against \(r\).
  \end{itemize}
\item
  \textbf{Interpret the Results}

  \begin{itemize}
  \tightlist
  \item
    Identify the critical parameter value(s) of \(r\) that cause a
    subcritical pitchfork bifurcation.\\
  \item
    Compare to the simpler subcritical system \(\dot{x} = r\,x + x^3\)
    to see how the additional \(-x^5\) term changes the bifurcation
    structure.
  \end{itemize}
\end{enumerate}

\begin{center}\rule{0.5\linewidth}{0.5pt}\end{center}

\end{document}
